% This paper is part of the single transits project.
% Copyright 2015 Dan Foreman-Mackey (NYU) and the co-authors listed below.
%
%  RULES OF THE GAME
%
%  * 80 characters
%  * line breaks at the ends of sentences
%  * eqnarrys ONLY
%  * ``light curve'' not ``light-curve'' or ``lightcurve''
%  * Do not put in any comments that might get tweeted by @OverheardOnAph
%    (or maybe do put in a few....)
%  * ``percent'' (not \%) is a unit, as is ppm, so 5~percent.
%  * that is all.
%

\documentclass[12pt,preprint]{aastex}

\pdfoutput=1

\usepackage{color,hyperref}
\definecolor{linkcolor}{rgb}{0,0,0.5}
\hypersetup{colorlinks=true,linkcolor=linkcolor,citecolor=linkcolor,
            filecolor=linkcolor,urlcolor=linkcolor}
\usepackage{url}
\usepackage{amssymb,amsmath}
\usepackage{subfigure}
\usepackage{booktabs}

\usepackage{natbib}
\bibliographystyle{apj}

\newcommand{\project}[1]{\textsl{#1}}
\newcommand{\kepler}{\project{Kepler}}
\newcommand{\KT}{\project{K2}}
\newcommand{\tess}{\project{TESS}}
\newcommand{\jwst}{\project{JWST}}
\newcommand{\terra}{\project{TERRA}}
\newcommand{\pdc}{\project{PDC}}
\newcommand{\license}{MIT License}

\newcommand{\paper}{\textsl{Article}}

\newcommand{\foreign}[1]{\emph{#1}}
\newcommand{\etal}{\foreign{et\,al.}}
\newcommand{\etc}{\foreign{etc.}}
\newcommand{\True}{\foreign{True}}
\newcommand{\Truth}{\foreign{Truth}}

\newcommand{\figref}[1]{\ref{fig:#1}}
\newcommand{\Fig}[1]{\figurename~\figref{#1}}
\newcommand{\fig}[1]{\Fig{#1}}
\newcommand{\figlabel}[1]{\label{fig:#1}}
\newcommand{\Tab}[1]{Table~\ref{tab:#1}}
\newcommand{\tab}[1]{\Tab{#1}}
\newcommand{\tablabel}[1]{\label{tab:#1}}
\newcommand{\Eq}[1]{Equation~(\ref{eq:#1})}
\newcommand{\eq}[1]{\Eq{#1}}
\newcommand{\eqalt}[1]{Equation~\ref{eq:#1}}
\newcommand{\eqlabel}[1]{\label{eq:#1}}
\newcommand{\sectionname}{Section}
\newcommand{\Sect}[1]{\sectionname~\ref{sect:#1}}
\newcommand{\sect}[1]{\Sect{#1}}
\newcommand{\sectalt}[1]{\ref{sect:#1}}
\newcommand{\App}[1]{Appendix~\ref{sect:#1}}
\newcommand{\app}[1]{\App{#1}}
\newcommand{\sectlabel}[1]{\label{sect:#1}}

\newcommand{\T}{\ensuremath{\mathrm{T}}}
\newcommand{\dd}{\ensuremath{\,\mathrm{d}}}
\newcommand{\bvec}[1]{{\ensuremath{\boldsymbol{#1}}}}

% TO DOS
\newcommand{\todo}[3]{{\color{#2}\emph{#1}: #3}}
\newcommand{\dfmtodo}[1]{\todo{DFM}{red}{#1}}
\newcommand{\hoggtodo}[1]{\todo{HOGG}{blue}{#1}}

\newcommand{\params}{\boldsymbol{\theta}}
\newcommand{\data}{\mathcal{D}}

\begin{document}

\title{%
    ABC stuffz
}

\newcommand{\nyu}{2}
\newcommand{\cds}{3}
\newcommand{\auck}{4}
\newcommand{\mpia}{5}
\author{%
    Daniel~Foreman-Mackey\altaffilmark{1,\nyu,\cds},
    Brendon~J.~Brewer\altaffilmark{\auck},
    David~W.~Hogg\altaffilmark{\nyu,\mpia,\cds},
    \etal
}
\altaffiltext{1}         {To whom correspondence should be addressed:
                          \url{danfm@nyu.edu}}
\altaffiltext{\nyu}      {Center for Cosmology and Particle Physics,
                          Department of Physics, New York University,
                          4 Washington Place, New York, NY, 10003, USA}
\altaffiltext{\cds}      {Center for Data Science, New York University,
                          726 Broadway, 7th Floor, New York, NY, 10003, USA}
\altaffiltext{\auck}     {Department of Statistics,
                          The University of Auckland, Private Bag 92019,
                          Auckland 1142, New Zealand}
\altaffiltext{\mpia}     {Max-Planck-Institut f\"ur Astronomie,
                          K\"onigstuhl 17, D-69117 Heidelberg, Germany}

\begin{abstract}

Blah blibby blah.

\end{abstract}

\keywords{%
words.
}

\section{Introduction}


With a prior distribution $p(\params)$ and sampling distribution
$p(\data|\params)$, we have defined the prior state of knowledge on the
joint space of possible parameter values and datasets:
\begin{eqnarray}
p(\params, \data) = p(\params)p(\data | \params).
\end{eqnarray}




\citep{Ballard:2014}.

\acknowledgments
It is a pleasure to thank
\ldots
for helpful contributions to the ideas and code presented here.
DFM and DWH were partially supported by the National Science Foundation
(grant IIS-1124794),
the National Aeronautics and Space Administration
(grant NNX12AI50G), and the Moore--Sloan Data Science Environment at NYU.

This research made use of the NASA \project{Astrophysics Data System} and the
NASA Exoplanet Archive.
The Archive is operated by the California Institute of Technology, under
contract with NASA under the Exoplanet Exploration Program.
This \paper\ includes data collected by the \kepler\ mission. Funding for the
\kepler\ mission is provided by the NASA Science Mission directorate.
We are grateful to the entire \kepler\ team, past and present.
Their tireless efforts were all essential to the tremendous success of the
mission and the successes of \KT, present and future.
These data were obtained from the Mikulski Archive for Space Telescopes
(MAST).
STScI is operated by the Association of Universities for Research in
Astronomy, Inc., under NASA contract NAS5-26555.
Support for MAST is provided by the NASA Office of Space Science via grant
NNX13AC07G and by other grants and contracts.

{\it Facilities:} \facility{Kepler}

\appendix

\section{Some appendix}

\clearpage
\bibliography{abc}
\clearpage


% \begin{figure}[p]
% \begin{center}
% \includegraphics{figures/pca.pdf}
% \end{center}
% \caption{%
% The top 10 eigen light curves (ELCs) generated by running principal component
% analysis on all the aperture photometry from Campaign~1.
% \figlabel{pca}}
% \end{figure}

\end{document}
